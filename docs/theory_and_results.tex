\documentclass[a4paper,12pt]{article}
%\documentclass[a4paper,10pt]{extarticle}
%\documentclass[a4paper,14pt]{extarticle} 

\usepackage{cmap}  % PDF search & text copy
%% Fonts
\usepackage[utf8x]{inputenc}

%% Language
	\usepackage[english, russian]{babel} 
	
%% Lines and margins
%% Double spacing
%\usepackage[nodisplayskipstretch]{setspace}
%\doublespacing  

%% Layout: choose one from below
%\usepackage[paper=a4paper,left=30mm, top=20mm, right=15mm, bottom=20mm]{geometry} 
\usepackage{fullpage} 

\setlength{\parindent}{0pt} % no indent at the new line

%% Math
\usepackage{amsmath,amssymb,empheq}
%\numberwithin{equation}{section} % in-section equation numbering
%\usepackage{breqn}
%\usepackage{mathtools} 		%  Remove equation numbers
%\mathtoolsset{showonlyrefs} %  from unreferenced equations

%% Graphics
\usepackage{graphicx}
%\usepackage[pdftex]{graphicx}
\usepackage{caption} % graphics caption
\usepackage{subcaption}
%\usepackage{epstopdf} % include eps graphics

% itemize options https://tex.stackexchange.com/a/2372
\usepackage{enumitem}
\setitemize[0]{leftmargin=10pt,itemindent=10pt} % label={-} 0 - change all levels
%\setitemize[1]{label={-}}
%\setitemize[0]{leftmargin=*}

%% \usepackage{markdown}  
% needs --shell-escape 
% use as \beging \end{markdown} - use markdown in between
% still exits with an error (because my build files are in a subfolder?)

%% Finally, give us PDF bookmarks
%\usepackage{color,hyperref}
%\definecolor{darkblue}{rgb}{0.0,0.0,0.3}
%\hypersetup{colorlinks,breaklinks,
%            linkcolor=darkblue,urlcolor=darkblue,
%            anchorcolor=darkblue,citecolor=darkblue}

%%% COMMANDS %%%

%% Math
\newcommand{\wt}{\widetilde} % tilde shortcut
\newcommand{\eps}{\varepsilon}
\newcommand{\bigO}{\mathcal{O}}

\newcommand{\vf}{\varphi}
\newcommand{\F}{\Phi} %{\mathbf{\Phi}}
\newcommand{\FP}{\Phi^*}%{\mathbf{\Phi^*}} % fixpoint 


\newcommand{\D}{\Delta}%{\mathbf{\Delta}} % Delta0
\newcommand{\Lmap}{\mathcal{L}} % Poincare map
\newcommand{\Lmat}{\mathbf{L}}  % Linearized Poincare map

\begin{document}

\section*{Notation}

\begin{itemize}
\item $\mathbf{k}$ - wave vector
\item $\mathbf{\Phi}= (\varphi_1, \varphi_2, .., \varphi_{N})^T $ - phase vector
\item $ \mathbf{Q(\Phi)}$ - vector of active driving forces
\item $ \mathbf{\Gamma}$ - friction coefficients NxN-matrix
\item $\mathbf{\Phi_k}$ - m-twist solution
\item $\phi(\mathbf{\Phi})$ - global phase, according to one of the definitions. *Currently mean phase is the default choice for global phase.*
\item $\mathcal{L}: H \rightarrow H$ - Poincare map
\item $\mathbf{\Phi^*} \in H$ - fix point of the Poincare map
\item $\mathbf{\Phi^{*}_k}$ - fix point close to the m-twist solution
\item $\mathbf{\Delta_0}$, $\mathbf{\Delta_1}$ - perturbation initial and after one cycle: $\mathcal{L}(\mathbf{\Phi^{*}} + \mathbf{\Delta_0}) = \mathbf{\Phi^{*}} + \mathbf{\Delta_1}$
\item $\mathbf{L} = \mathrm{D}\mathcal{L}(\mathbf{\Phi^*})$ linearized Poincare map at a fixed point (matrix). In code `Lmat`
\item $\mathbf{L} = e^\mathbf{\Lambda}; \quad \mathbf{\Lambda} = \log \mathbf{L} $  logarithm of the linearized Poincare map. In code `Lmat\_log`
\item $\lambda_j$ - eigenvalues of $\Lambda$
\end{itemize}

**Conflicts:**

\begin{itemize}

\item Fixpoint notation and complex conjugation.
\item Once $\delta$ and $\D$ are taken - how to denote a difference of some values?
\item global phase and components of $\mathbf{\Phi}$
\item $d(\Phi)$ in procedure to find fixpoint and $d$ - as a distance between cilia. Change $d$ to $a$ - spacing between cilia?
\item TODO: $\mathbf{\Delta_0}$ -> $\mathbf{\Delta}_0$ 

\end{itemize}


%**Fill the tables to see which letters I can still use?**

%|         |   |         |   |
%|---------|---|---------|---|
%| a       |   | A       |   |
%| b       |   | B       |   |
%| c       |   | C       |   |
%| d       |   | D       |   |
%| e       | basis  | E       | Ben's distance from mtwist to fixpoint  |
%| f       |   | F       |   |
%| g       |   | G       |   |
%| h       |   | H       |   |
%| i       | idx  | I       |   |
%| j       | idx  | J       |   |
%| k       | idx, mtwist number, dual space vector  | K       |   |
%| l       | idx | L       |   |
%| m       |   | M       |   |
%| n       | $n_x$, $n_y$ - num cilia in 1 direction  | N       |  num cilia |
%| o       |   | O       |   |
%| p       |   | P       |   |
%| q       |   | Q       |  gen force |
%| r       |   | R       |   |
%| s       |   | S       |   |
%| t       | time, translation vec  | T       | period  |
%| u       |   | U       |   |
%| v       |   | V       |   |
%| w       |   | W       |   |
%| x       | coords  | X       |   |
%| y       | coords  | Y       |   |
%| z</pre> | coords  | Z</pre> |   |

\section*{Geometry}

*Unit vectors*
$$
\mathbf{e}_1 =\left( \begin{array}{c} 1 \\ 0 \end{array} \right), \quad
\mathbf{e}_2 =\left( \begin{array}{c} \frac{1}{2} \\  \frac{\sqrt{3}}{2} \end{array} \right).
$$

*Position vectors*
$$
\mathbf{x}_{n,m} = n\,a\,\mathbf{e}_1 +  m\,a\,\mathbf{e}_2
$$

*Honeycomb lattice*
$$
\mathcal{H} = \{ \mathbf{x}_{n,m} | 2n+m \equiv 0 \text{ or } 2 \text{ mod } 3\}
$$

*Triangular lattice*
$$
\mathcal{T} = \{ \mathbf{x}_{n,m} | n,m \in \mathrm{Z} \}
$$

\section*{m-twist solutions}

*Wave vector of metachronal wave consistent with periodic boundary condition*
$$ \mathbf{k} = \frac{a_1}{L_1}\,k_1\,\mathbf{a}_1^* + \frac{a_2}{L_2}\,k_2\,\mathbf{a}_2^*, $$
where $k_1,k_2\in{Z}$ and $a_1=a$, $a_2 = \sqrt{3} a/2$

*Meta-chronal wave solutions:*

Phase-space vector
$ \Phi (t) = (\varphi_1, \varphi_2, .., \varphi_{N})^T $
with components
$$
\varphi_i(t) = \varphi(t) - \mathbf{k} \cdot \mathbf{x}_i,
$$
for some global phase $\varphi(t)$.
We expect $\dot{\varphi}\approx\omega_0$, but only approximately, since the calibration of active driving forces does not take into account the (weak) hydrodynamic interactions.

\subsection*{Dynamic equation from force-balance equation}

$$ \dot{\Phi} = \mathbf{\Gamma}^{-1}(\Phi)Q(\Phi)  $$
This linear system can be solved directly in Python.


\subsection*{from hydrodynamic simulations}

*Pair-wise coupling functions:*
$$ \Gamma_{ij}(\Phi) = 0 \quad \text{if not neighbours} $$
$$ \Gamma_{ij}(\Phi) = \Gamma_{12}^{\mathrm{(loc)}}(\varphi_i,\varphi_j; d, \psi) \quad \text{if neighbours} $$
where the translation vector pointing from cilium $i$ to cilium $j$ reads
$$[n(j)-n(i)]\,\mathbf{e}_1 + [m(j)-m(i)]\,\mathbf{e}_2 = d \left( \begin{array}{c} \sin\psi \\ \cos\psi \end{array} \right).$$
*Self-friction:*

$$ \Gamma_{ii}(\Phi) = \Gamma_{11}^{\mathrm{(loc)}}(\varphi_i, \varphi_j; d, \psi) $$

Note, self-friction $\Gamma_{11}^{\mathrm{(loc)}}(\varphi_i, \varphi_j; d, \psi) $ dependence on $\varphi_j$, $d$, $\psi$ is so weak, that we can take
$\Gamma_{11}^{\mathrm{(loc)}}(\varphi_i, \varphi_j; d, \psi)= \Gamma_{11}^{\mathrm{(loc)}}(\varphi_i)$



*Active driving forces (simplest case: calibration for isolated cilium):*
$$ Q_{i}(\Phi) = \Gamma_{11}^{\mathrm{(loc)}}(\varphi_i) \omega_0 $$


\subsection*{Global Phase}

\subsubsection*{Naive phase}
For some index $j$, $\varphi = \mathbf{\varphi}_j$

\subsubsection*{Circular average phase}

We define order parameters
%[[Kuramoto, 1984]](https://link.springer.com/chapter/10.1007/978-3-642-69689-3_7)  [[Strogatz, 2000]](https://www.sciencedirect.com/science/article/pii/S0167278900000944)
for each metachronal wave vector $\mathbf{k}$

$$S_\mathbf{k} = \left| \frac{1}{N}\sum_j \exp\left[ i ( \varphi_j - \varphi_{\mathbf{k},j} ) \right] \right|,$$

where $\varphi_{\mathbf{k},j}$ - j-th component of $\Phi_\mathbf{k}$, m-twist solution defined by vector $\mathbf{k}$.

**Global phase:**

For a fixed $\mathbf{k}$
$$\varphi(t; \mathbf{k}) = \arg \left( \frac{1}{N}\sum_j \exp\left[ i ( \varphi_j(t) - \varphi_{\mathbf{k},j} ) \right] \right) $$

- Only has meaning when $S_\mathbf{k}$ is close to 1.

\subsubsection*{Mean phase}

Define the global phase as algebraic mean of all phases
$$
\varphi =  \frac{1}{N}\sum_{j=1}^{N} \varphi_j
$$

Pros
- Additive: if $\mathbf{\Phi} = \mathbf{\Phi_0}+ \mathbf{\Phi_1} $, then $\varphi = \varphi_0 + \varphi_1$
- As an implication, it is independent of which initial condition we consider.
- Poincare surface will be a plane, defined by normal $(1,1,1,1..,1)$.
  [A language of quotient vector space could be useful; just keep this reference here for now http://mathworld.wolfram.com/QuotientVectorSpace.html]
  (quotient = частное)

Cons
- $\varphi_j$ can't jump from  $2 \pi$ to $0$, otherwise the global phase will make a jump by $- \frac{2 \pi}{N}$.

\section*{Poicnare map for a m-twist solution}
\subsection*{Poincare map and limit cycle}
We consider a Poincare section $H$ defined by

$$
H =\{ \Phi : \varphi(\Phi) \equiv \varphi(\Phi_0) \mod 2\pi \}.
$$

- $\varphi$ denotes the global phase.
- $\Phi_0$ - phase vector at initial time $t_0$.
- $H$ is a $(N-1)$-dimensional hypersurface in $N$-dimensional phase space.
  - If $\varphi$ - mean phase, $H$ is a hyperplane.

For each of m-twists solutions, we anticipate a corresponding limit cycle $C_\mathbf{k}$, piercing $H$ close to $\Phi_k$ i.e.

$$
C_\mathbf{k} \cap H_0 = \Phi_\mathbf{k} + \mathbf{E}^*,
$$

with a small correction vector $\mathbf{E}^*$.
The reason for the small correction $\mathbf{E}^*$ is that the calibration of active driving forces causes small phase-dependent variations of the instantaneous phase speed *(a cilium will speed up at one part of the beat cycle, and slow down at another; but then the phase difference will vary during the cycle, therefore we don't get a perfect m-twist, which has constant phase difference)*.

Therefore, the first step is to find $\Phi^*_\mathbf{k}=\Phi_\mathbf{k} + \mathbf{E}^*$.

We define **Poincare map** $\mathcal{L}: H \rightarrow H$ as
 $$\mathcal{L}(\Phi_0) = \Phi_1,$$
where
- $\Phi_0 = \Phi(t_0)$ - some initial phase. *Note that for any $\Phi_0$ we can define a Poincare section as defined above.*
- $\Phi_1=\Phi(t_1)$ - where $t_1$ is the next time when our phase trajectory hits the Poincare section.

Then  $\Phi^*_\mathbf{k}$ is a fixpoint of Poincare map

**TODO:** explain why it cannot be another limit cycle in Poincare plane.

\subsubsection*{Procedure to find the fixpoint $\Phi^*$}

By definition, the fixpoint is such a point that
$$
\mathcal{L}(\Phi^*) = \Phi^*
$$
We define (**TODO:** this function can be useful in other places - keep this notation?)
$$
D(\Phi) = \mathcal{L}(\Phi) - \Phi
$$
and
$$
d(\Phi) = \lVert D(\Phi) \rVert ^ 2
$$
Function $d(\Phi)$ is the squared distance between a phase vector and its Poincare map image.

Properties of norm imply that
- $d(\Phi) \geq 0$
- $d(\Phi)=0 \iff \Phi = \Phi^*$ - a fixed point

Therefore, to find the fixpoint $\Phi^*_{\mathbf{k}}$, we numerically find a minimum of function $d$ with initial condition in $\Phi_{\mathbf{k}}$. Used library function `scipy.optimize.minimize` with method `BFGS`.


\subsection*{ Linearized Poincare map}

We use linear stability analysis to study stability of limit cycles. Consider a fixpoint of Poincare map $\Phi^*$ (corresponds to a limit cycle) and apply a small perturbation $\_0$.

$$\mathcal{L}(\mathbf{\Phi^{*}} + \D_0) = \mathbf{\Phi^{*}} + \D_1$$

Power expansion in vicinity of the fixpoint yields

$$\mathcal{L}(\mathbf{\Phi^{*}} + \D_0) = \mathbf{\Phi^{*}} +  \mathrm{D}\mathcal{L}(\mathbf{\Phi^*}) \D_0 + \mathcal{O}(\lVert\ \D_0 \rVert^2) $$

where $\mathrm{D}\mathcal{L}(\mathbf{\Phi^*})$ is linear contribution (represented by a matrix), also known as linearized Poincare map [reference].

Further let's use short notation $\mathbf{L}$ = $\mathrm{D}\mathcal{L}(\mathbf{\Phi^*})$. In code `Lmat`

TODO: procedure to find L

TODO:

$\mathbf{L} = e^\mathbf{\Lambda}; \quad \mathbf{\Lambda} = \log \mathbf{L} $


- $\mathbf{\Lambda}$ - is not a symmetrical matrix.
- $ \log \mathbf{L}$ is close to $\mathbf{L} - \mathbf{I}$, but numerically I found that real part of eigenvalue can differ up to 20% (although abs value differs only by 2%)

\subsection*{Linearized Poincare map: another vision}
- Since Poincare section is N-1 dimensional hypersurface, the linearized map is represented by N-1 to N-1 matrix.
- We construct it by perturbing the fixpoint with N-1 perturbations.
  - Those are such perturbations, that the perturbed state still lies in Poincare section.
  - To work with N to N matrix we add Nth perturbation $\D_0 = (1,...1)$ - normal to the Poincare plane and declare that $\D_1 = \D_0$.

   That will add another eigenvalue $\lambda=1$ and other eigenvalues and eigenvectors will remain unchanged.


\section*{Numerical inaccuracy discussion}

\subsection*{Inaccuracy in fixpoint}

\subsubsection*{Fixpoint}


\begin{itshape}
{\small 
Ben suggested a different notation:
\begin{itemize}
\item  difference between real fixpoint and approximate: $\wt \D_0$
\item  After one cycle distance to the real fixpoint: $\wt \D_1$
\item Then always use $\D_0$, $\D_1$ like before

\end{itemize}
Downsides:

\begin{itemize}
\item New $\wt \D_1$ is not $\D_*$, but we can only compute latter.
\item In code we subtract numerical fixpoint, but we should  subtract its Poincare map image.
\end{itemize}

}
\end{itshape}

We find fixpoint by optimization procedure [defined above] up to a tolerance, specified in code. Two tolerances are involved: (i) solver tolerance and (ii) minimizer tolerance. After some minimal testing, those were taken to be equal and further in this section both are referred simply as tolerance.

Suppose we found numerically a fixpoint $\wt \Phi^*$, but it is not perfect and $ \mathcal{L}(\wt \FP) \neq \wt \FP $:

$$ \mathcal{L}(\wt \FP) = \wt \FP + \D_*$$

This means, that real fixpoint is somewhere else:

$$\wt \FP= \FP + \wt{\D}$$

If we consider linear approximation in vicinity of $\FP$


$$ \mathcal{L}(\wt \FP) = \FP + \mathbf{L}  \wt{\D}$$

$$ \wt \FP + \D_* = \FP + \mathbf{L}  \wt{\D}$$

$$  \FP + \wt{\D} + \D_* = \FP + \mathbf{L}  \wt{\D}$$

Therefore

\begin{equation}
\D_* = ( \mathbf{L} - \mathbf{I} ) \wt{\D} + \mathcal{O}(\lVert \wt{\D} \rVert^2)
\label{eqn:delta_star}
\end{equation}



Note
\begin{itemize}

\item We don't know real $\mathbf{L}$ (only approximate at $\wt \FP$)
\item We don't know distance to the real fixpoint $\wt{\D}$.
\item We can, however, calculate $\D_*$.

\end{itemize}

Let's denote
 $\varepsilon_{F}
 = \lVert \D_* \rVert
 =\lVert \mathcal{L}(\wt{\FP}) - \wt{\FP}  \rVert$
and $\delta_F = \lVert \wt{\D} \rVert $. Then

$$
\lVert \D_* \rVert = \lVert ( \mathbf{L} - \mathbf{I} ) \wt{\D} \rVert
$$


$$
\min|\lambda| \delta_F
\leq \varepsilon_{F}
\leq \max|\lambda| \delta_F
$$

$$
\varepsilon_{F} \sim |\lambda| \delta_F
$$



\subsubsection*{Perturbed state}
Let's consider a perturbed (approximate) fixpoint. Fixpoint error will add up to the perturbation:

$$
\mathcal{L}(\wt{\Phi}^* + \D_0)
= \mathcal{L}(\FP + \wt  \D +  \D_0 )
= \FP +  \mathbf{L}  ( \wt{\D} +  \D_0 )
$$
The left side is equal to the initial state plus deviation

$$
\mathcal{L}(\wt \FP  + \D_0)
= \wt \FP + \D_1
= \FP + \wt{\D}  + \D_1
$$

And combining these two equalities

$$
\D_1
=  \mathbf{L}   \D_0  + (\mathbf{L} - \mathbf{I}) \wt{\D}
$$

$$
\D_1 - \D_0
=  (\mathbf{L} - \mathbf{I})  \D_0  + (\mathbf{L} - \mathbf{I}) \wt \D
$$

$$
\D_1 - \D_0
=  (\mathbf{L} - \mathbf{I})  \D_0  + \D_*
$$

Now let's estimate norms

$$
\lVert \D_1 - \D_0 \rVert
\sim
|\lambda|(\delta_0 + \delta_F)
\sim
|\lambda|\delta_0 + \varepsilon_F
$$

$$
\frac{ \lVert \D_* \rVert }  {\lVert \D_1 - \D_0 \rVert }
\sim
\frac{ \varepsilon_F } { |\lambda| \delta_0 + \varepsilon_F}
$$

If $|\lambda| \delta_0 >>  \varepsilon_{F}:$

$$
\frac{ \lVert \D_* \rVert }  {\lVert \D_1 - \D_0 \rVert }
\sim
\frac{ \varepsilon_F } { |\lambda| \delta_0}
\sim
\frac{ \delta_F } {  \delta_0 }
$$


\paragraph*{Conclusions}
- We want to make sure that $\frac{ \lVert \D_* \rVert }  {\lVert \D_1 - \D_0 \rVert } $ is small. Otherwise, $\D_1 - \D_0$ is significantly affected by $\D_*$ contribution, and we will have big error when we calculate matrix $\mathbf{L}$. TODO: quantify

- $\D_1 - \D_0$ has a constant contribution equal to $\D_*$.
- Note that $\D_0$, $\D_1$ - simulation input and outputs, so that's something we can directly obtain.
- $ \D_* $ - is something that we can measure, and can control indirectly, by tuning
  algorithm tolerance.
- Therefore, we must make sure that
  - Test at $N=6$ showed that this ratio is around or below $10^{-2}$ if fixpoint tolerance is $10^{-8}$.

- Also, just for fun, let's note

$$
 \mathcal{L}(\wt \FP  + \D_0) - \mathcal{L}( \FP  + \D_0) = D\mathcal{L} \vert_{\wt \FP  + \D_0} \wt \D + \mathcal{O}(\delta_F^2)
$$


\subsubsection*{Linearized map}
Expand Poincare map $\mathcal{L}$ in vicinity of $\FP$

$$
\mathcal{L}(\wt \FP+ \D_0)
= \FP +  \mathbf{L}  ( \wt \D +  \D_0 ) + \mathcal{O}(( \delta_F + \delta_0)^2)
$$ % = ... + \mathcal{O}(\delta_F^2, \delta_F \delta_0,  \delta_0^2)

Do the same in vicinity of $\wt \FP$

$$
\mathcal{L}(\wt \FP + \D_0)
= \mathcal{L}(\wt \FP) +  \wt{\mathbf{L}} \D_0 + \mathcal{O}(\delta_0^2) =
$$

$$
=  \FP + \Lmat \wt \D + \mathcal{O}(\delta_F^2)  +  \wt{\mathbf{L}} \D_0 + \mathcal{O}(\delta_0^2)
$$%= \FP + \wt \D + \D_* +  \wt{\mathbf{L}} \D_0 + \mathcal{O}(\delta_0^2)


Combining those two

$$
\mathbf{L}  ( \wt \D +  \D_0 ) + \mathcal{O}(( \delta_F + \delta_0)^2)
=
\Lmat \wt \D + \mathcal{O}(\delta_F^2)  +  \wt{\mathbf{L}} \D_0 + \mathcal{O}(\delta_0^2)
$$


$$
\mathbf{L} \wt \D 
=
 \wt{\mathbf{L}} \D_0 + \mathcal{O}(\delta_0^2) + \mathcal{O}(\delta_F^2) + \mathcal{O}(( \delta_F + \delta_0)^2)
$$

$$
(\mathbf{L} -  \wt{\mathbf{L}}) \D_0 
=
\mathcal{O}(\delta_0^2) + \mathcal{O}(\delta_F^2) + \mathcal{O}(\delta_F \delta_0)
$$

Some of $\mathcal{O}$ must annihilate. Which ones?

Let's take a new perturbation: $2 \D_0$, then 

\textbf{TODO:} represent big Os as powerseries? Claim that terms which scale differently that the left side must annihilate. 

Expected result:
$$
\mathbf{L} -  \wt{\mathbf{L}}
=
 \mathcal{O}(\delta_F)
$$

\subsubsection*{Approximate $\Lmap(\wt \FP)$ as $\wt \FP$}

\begin{equation}
\Lmap(\wt \FP) = \FP + \Lmat \wt \D + \mathcal{O}(\delta_F ^2)
\end{equation}
Or
\begin{equation}
\Lmap(\wt \FP) - \FP = \mathcal{O}(|\lambda| \delta_F)
\end{equation}


\subsubsection*{Numerical estimations}

If $N = 6$, fixpoint with $ tol  =10 ^{-8}$:

$|\lambda| = 10 ^ {-2}$ - $10 ^ {-3}$

$\varepsilon_F = 10 ^ {-6}$ - $10^{-7}$

Therefore $\delta_F \sim 10 ^{ -4}$

If $\delta_0 = 10 ^{-3}$, $\frac{ \delta_F } {  \delta_0 } = 10^{-1}$, and indeed,
$ \frac{ \lVert \D_* \rVert }  {\lVert \D_1 - \D_0 \rVert }$ lie in range $2 * 10^{-2} - 10 ^{-3} $




\newpage
\part*{Simulations}


\subsection*{Eigenvectors}

\begin{itemize}

\item Stored as \textbf{columns} of NxN array \textit{evecs}
\item Normalized
\item Not orthogonal in general, but a lot of them \emph{are} orthogonal
\item Most of eigenvectors are complex, they come in pairs: eigenvector and its complex-conjugates, eigenvalue and its complex conjugate [algebra]
\item Perturbation made of eigenvector and its complex conjugate will develop after a cycle as predicted to linear theory up to precision of $10^{-4} - 10^{ -5}$.

 That's not very small, but it remains as small if I increase $\delta_0$ to $10 ^{ -1}$.
 
\item Imaginary part of eigenvalue gives only a small contribution ($e^{0.02 i} \approx 0.9998 + 0.02 i$).
\item Most of eigenvectors have their components lying on a circle.

\end{itemize}

\subsubsection*{ Eigenvector decomposition}
Let's consider an eigenvector $\D$. We can decompose it into basis of complex exponents of m-twists (coefficients are essentially multidimensional discrete Fourier transform output):
$$
\D = \sum_{\mathbf{k}} d_k e^{i \Phi_\mathbf{k}}.
$$
Observed (1D carpet,\textit{try09v2}, \textit{try12v2}) that in fact only one component gives major contribution $\mathbf{k}=\mathbf{k_0}$,

$$
\D = d_{\mathbf{k_0}} e^{i \Phi_\mathbf{k_0}}  + \mathbf{R_1},
$$

where $\mathbf{R_1} = \sum_{\mathbf{\mathbf{k} \neq \mathbf{k_0}}} d_k e^{i \Phi_\mathbf{k}}$ - residual.

In most of the cases $\lVert \mathbf{R_2} \rVert < 0.04$. It is observed [\textit{try12v2}, ev vs k], that if eigenvalues are small by absolute magnitude ($\max|\lambda| \sim  10^{-3}$), residual can go up: $\lVert \mathbf{R_2} \rVert$ up to $0.5$. Taking just one additional term would lower the residual back to $0.04$. Either way, correctness of numerics in case of very small eigenvalues is under question, further let's neglect this case (which also happens usually only on the boundary between stable and unstable regions).

Representation of the eigenvector in the form
$$
\D \approx d_{\mathbf{k_0}} e^{i \Phi_\mathbf{k_0}}
$$
means that eigenvector components all lie on a circle in the complex plane.

\subsubsection*{Purely real eigenvalues}

If eigenvalue is real, then eigenvector components lie on the real axis line instead of a circle. It can probably decomposed into cos/sin of m-twists. [Under investigation: pure sine coupling `try14`]

\subsubsection*{Real perturbation out of complex eigenvectors}

We can construct real perturbation based on complex eigenvectors:
$$
\operatorname{Re}(\D)
= \frac{1}{2} ( \D + \D^*) 
\approx  \operatorname{Re}(d_{\mathbf{k_0}}) \cos(\Phi_{\mathbf{k_0}}) - \operatorname{Im}(d_{\mathbf{k_0}}) \sin(\Phi_{\mathbf{k_0}})
$$

$$
\operatorname{Im}(\D)
= \frac{1}{2} ( \D - \D^*)
\approx  \operatorname{Im}(d_{\mathbf{k_0}}) \cos(\Phi_{\mathbf{k_0}}) + \operatorname{Re}(d_{\mathbf{k_0}}) \sin(\Phi_{\mathbf{k_0}})
$$

\emph{TODO: can I force coefficient d to be always real?
}

After one cycle:
$$
\D_0 = \frac{1}{2} (\D + \D^*) 
$$
If $d_{\mathbf{k}_0}$ is real:
$$
\D_0 =  d_{\mathbf{k_0}} \cos(\Phi_{\mathbf{k_0}})
$$

$$
\D_1 - \D_0 = \lambda \D + \lambda^* \D^* = \operatorname{Re}(\lambda)d_{\mathbf{k_0}} \cos(\Phi_{\mathbf{k_0}}) - \operatorname{Im}(\lambda)  d_{\mathbf{k_0}} \sin(\Phi_{\mathbf{k_0}})
$$


\section*{Cilia carpet simulation}
[\textit{try11}]




\section*{Toy model: chain of oscillators with trigonometrical coupling}
% Command for this section
\newcommand{\fr}[2]{\frac{2 \pi #1} {N} #2 } %shortcut for 2pi k / N * j
\newcommand{\si}[2]{\sin \left( \fr{#1}{#2} \right)} %shortcut for sin, cos, exp
\newcommand{\co}[2]{\sin \left( \fr{#1}{#2} \right)} 
\newcommand{\ex}[2]{e ^{i \fr{#1}{#2}}}


\subsection*{Sine coupling}


Consider a chain of $N$ oscillators with equal beating frequency $\omega_0$, simple sinusoidal coupling and periodic boundary conditions (below we assume index $j$ modulo $N$).

\begin{equation}
\dot \varphi_j = \omega_0 ( 1 + \lambda_s \sin(\varphi_j - \varphi_{j-1})
+ \lambda_s \sin(\varphi_j - \varphi_{j-1})
\label{eqn:dphidt-sine}
\end{equation}

Limit cycles are perfect m-twists [substitute to prove derivative = const] [call it k-twist??]

$$
\varphi_j(t) = \omega_0 t - \fr{k}{j}, \quad j=1..N
$$

From our simulations we guess that eigenvectors will be a sine or cosine of an m-twist.

To prove it, let's perturb the limit cycle and find how this perturbation will develop in time
$$
\varphi_j \big\rvert_{t=0} = - \fr{k}{j} -  \eps \si{m}{j}, \quad j=1..N
$$
\textit{Note that we perturb with a sine of m-twist $\fr{m}{j}$ and move the minus outside sign.
}

Substitute that into \eqref{eqn:dphidt-sine}

\begin{equation}
\dot \varphi_j = .. \text{  See scanned pdf notes}
\end{equation}

\begin{equation}
\dot \varphi_j = 
\omega_0 \left( 1 +  2 \lambda_s  \co{k}{} \left(1 - \co{m}{} \right) \left( - \eps \si{m}{j} \right) \right)  
+ \bigO(\lambda_s \eps ^2 \omega_0)
\end{equation}

The perturbation grows or decays with rate (Lyapunov exponent)
$$
A = 2 \lambda_s  \co{k}{} \left(1 - \co{m}{} \right)
$$

Group frequency is the same as frequency of a single oscillator. Period $T = \frac{2 \pi}{\omega_0}$.

Poincare map eigenvalue
$$
\lambda = e^{A T} - 1 
= \exp \left( 4 \pi \lambda_s  \co{k}{} \left(1 - \co{m}{} \right) \right) - 1 
$$

In case of sufficiently weak coupling \textit{[but to be safe it's better to stay with a general formula, it doesn't give an advantage to move to a simplified formula]}
$$
\lambda 
= 4 \pi \lambda_s  \co{k}{} \left(1 - \co{m}{} \right)  + \bigO(\lambda_s^2)
$$

[TODO: EIGENSPECTRUM PLOTS AND CONCLUSIONS]

\subsection*{Sine plus cosine coupling}
\begin{equation}
\dot \varphi_j = \omega_0 \left( 1 + \lambda_s \sin(\varphi_j - \varphi_{j-1})
+ \lambda_s \sin(\varphi_j - \varphi_{j-1})
+ \lambda_c \cos(\varphi_j - \varphi_{j-1})
+ \lambda_c \cos(\varphi_j - \varphi_{j-1}) \right)
\label{eqn:dphidt-sin-cos}
\end{equation}

Now consider a complex perturbation

$$
\varphi_j \big\rvert_{t=0} = - \fr{k}{j} + \eps \exp( - \fr{m}{j})
$$

Again substitute into derivative [see pdf notes 2019-08] and end up with

$$
\dot \varphi_j = \omega + A \eps \exp( - \fr{m}{j}) + \bigO(\lambda_0 \eps ^ 2 \omega_0),
$$
where $\lambda_0 = \max(|\lambda_s|, |\lambda_c|)$ - measure of coupling strength,

$\omega$ - effective frequency
$$
 \omega = \omega_0 (1 + \lambda_c \co{k}{}),
$$

$A$ - perturbation change rate; Lyapunov exponent

$$ 
A = 
 2 \lambda_s \omega_0 \co{k}{} \left(1 - \co{m}{} \right)
- 2 i \lambda_c \omega_0 \si{k}{} \si{m}{}
$$

Poincare map eigenvalue ($L - I$)

$$ 
\lambda = e^{A T} - 1,
$$
where $T = \frac{2 \pi}{\omega}$

TODO: add plots and analysis

\end{document}

